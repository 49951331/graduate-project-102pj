%
% 使用 utf-8 編碼
% v1.0 (2013.07.17)
% do not change the content of this file
% unless the thesis layout rule is changed
% 無須修改本檔內容,除非校方修改了
% 封面、書名頁、中文摘要、英文摘要、誌謝、目錄、表目錄、圖目錄、程式碼目錄符號說明
% 等頁之格式

% make the line spacing in effect
\renewcommand{\baselinestretch}{\mybaselinestretch}
\large % it needs a font size changing command to be effective

% default variables definitions
% 此處預設值,請依照需求更改此處
\newcommand\cTitle{學生證照管理系統}
%\newcommand\eTitle{MY THESIS TITLE}
\newcommand\advisorCnameA{林慶昌 博士}
%\newcommand\advisorEnameA{Dr. Ching-Chang Lin}
\newcommand\univCname{臺北城市大學}
%\newcommand\univEname{Taipei Chengshih University of Science and Technology}
\newcommand\deptCname{資訊管理系}
%\newcommand\deptEname{Information Management}
\newcommand\cYear{102}
\newcommand\cMonth{12}
\newcommand\cDay{20}
% 設定班級,學生姓名
\newcommand\className{四資三忠}
\newcommand\idA{49951363} \newcommand\userA{陳信華(組長)}
\newcommand\idB{49951308} \newcommand\userB{曹詩敏}
\newcommand\idC{49951329} \newcommand\userC{陳廉忠}
\newcommand\idD{49951331} \newcommand\userD{黃柏崴}
\newcommand\idE{49951362} \newcommand\userE{林揚評}

% 使用 hyperref 在 pdf 簡介欄裡填入相關資料
\ifx\hypersetup\undefined
	\relax  % do nothing
\else
	\hypersetup{
	pdftitle=\cTitle,
%	pdfauthor=\myCname
}
\fi
	

\newcommand\itsempty{}
%%%%%%%%%%%%%%%%%%%%%%%%%%%%%%%
%       TPCU cover 封面
%%%%%%%%%%%%%%%%%%%%%%%%%%%%%%%
%
\begin{titlepage}
% no page number
% next page will be page 1

% aligned to the center of the page
\begin{center}
% font size (relative to 12 pt):
% \large (14pt) < \Large (18pt) < \LARGE (20pt) < \huge (24pt)< \Huge (24 pt)
%
\makebox[13cm][s]{\Huge{\univCname\deptCname}}\\  %顯示中文校名
\vspace{2cm}
%\makebox[12cm][s]{\Huge{\deptCname}}\\ %顯示中文系所名
%\vspace{1.5cm}
\makebox[8cm][s]{\Huge{\cYear 學年度專題製作}}\\
\vspace{3cm}
%
% Set the line spacing to single for the titles (to compress the lines)
\renewcommand{\baselinestretch}{1}   %行距 1 倍
%\large % it needs a font size changing command to be effective
%\Large{\cTitle}\\  % 中文題目
\fontsize{20}{22}\selectfont{\cTitle} \\
\vspace{3cm}
%\Large{\eTitle}\\ %英文題目
%\fontsize{20}{22}\selectfont{\eTitle} \\

指導老師: \advisorCnameA \\[2cm]

%\hspace{-8.4cm} \makebox[3cm][s]{班級:四資三篤}  \\[0.3cm]
%學生:學號:\idA 姓名:\userA \\[0.3cm]
%   學號:\idB 姓名:\userB \\[0.3cm]
%   學號:\idC 姓名:\userC \\[0.3cm]
%   學號:\idD 姓名:\userD \\[0.3cm]
%   學號:\idE 姓名:\userE \\
\begin{tabular}{ll}
班級:\className & \\[0.3cm]
學號:\idA & 姓名:\userA \\[0.3cm]
學號:\idB & 姓名:\userB \\[0.3cm]
學號:\idC & 姓名:\userC \\[0.3cm]
學號:\idD & 姓名:\userD \\[0.3cm]
學號:\idE & 姓名:\userE \\[0.3cm]
\end{tabular}

%\vspace{4cm}
% \makebox is a text box with specified width;
% option s: stretch
% use \makebox to make sure
\vfill
\makebox[10cm][s]{\Large{中華民國 \cYear 年 \cMonth 月 \cDay 日}}%
%
\end{center}
% Resume the line spacing to the desired setting
\renewcommand{\baselinestretch}{\mybaselinestretch}   %恢復原設定
% it needs a font size changing command to be effective
% restore the font size to normal
\normalsize
\end{titlepage}

%%%%%%%%%%%%%%%%%%%%%%%%%%%%%%%
%  TPCU Side Page 論文的書背
%%%%%%%%%%%%%%%%%%%%%%%%%%%%%%%
%
% Command to generate the side page.
\setCJKfamilyfont{heivert}[Vertical=RotatedGlyphs]{微軟正黑體}
% 解決 中英文混何直排時,英文太靠右的問題
\newcommand*\CJKmovesymbol[1]{\raise.35em\hbox{#1}}
%\newcommand*\CJKmovesymbol2[1]{\raise.65em\hbox{#1}}
\newcommand*\CJKmove{\punctstyle{plain}% do not modify the spacing between punctuations
  \let\CJKsymbol\CJKmovesymbol
  \let\CJKpunctsymbol\CJKsymbol
}

\begin{center}
\thispagestyle{empty}
\CJKfamily{heivert}\CJKmove
\fontsize{14}{14}\selectfont
\noindent\rotatebox{-90}{
\begin{tabular}{
	m{2cm}m{0.1cm}m{14cm}m{0.1cm}m{2.5cm}m{0.1cm}m{2.5cm}}
\rotatebox{90}{\hspace{-0.45em}\cYear} 學年度 & & 
\cTitle & & \deptCname & & 四技部
\end{tabular}
}
\end{center}

%% 從摘要到本文之前的部份以小寫羅馬數字印頁碼
% 但是從「書名頁」(但不印頁碼) 就開始計算
\setcounter{page}{1}
\pagenumbering{roman}
% 目前送出空白頁
\newpage

%%%%%%%%%%%%%%%%%%%%%%%%%%%%%%%
%       論文口試委員審定書 (計頁碼,但不印頁碼)
%%%%%%%%%%%%%%%%%%%%%%%%%%%%%%%
%
% insert the printed standard form when the thesis is ready to bind
% 在口試完成後,再將已簽名的審定書放入以便裝訂
% create an entry in table of contents for 審定書

\thispagestyle{EmptyWaterMarkPage}  % 無 header,但在浮水印模式下會有浮水印
\phantomsection % for hyperref to register this
\addcontentsline{toc}{chapter}{\nameCommitteeForm}%
\begin{center}
\fontsize{20}{14}\selectfont
% font size (relative to 12 pt):
% \large (14pt) < \Large (18pt) < \LARGE (20pt) < \huge (24pt)< \Huge (24 pt)
%
\makebox[13cm][s]{\Huge{\univCname\deptCname}}\\  %顯示中文校名
\vspace{1cm}
%\makebox[12cm][s]{\Huge{\deptCname}}\\ %顯示中文系所名
%\vspace{1cm}
\makebox[8cm][s]{\Huge{\cYear 學年度專題製作}}\\
\vspace{1cm}
%
% Set the line spacing to single for the titles (to compress the lines)
\renewcommand{\baselinestretch}{1}   %行距 1 倍
%\hspace{-8.4cm} \makebox[3cm][s]{班級:四資三篤}  \\[0.3cm]
%學生:學號:\idA 姓名:\userA \\[0.3cm]
%   學號:\idB 姓名:\userB \\[0.3cm]
%   學號:\idC 姓名:\userC \\[0.3cm]
%   學號:\idD 姓名:\userD \\[0.3cm]
%   學號:\idE 姓名:\userE \\[0.3cm]

\begin{tabular}{ll}
班級:\className & \\[0.3cm]
學號:\idA & 姓名:\userA \\[0.3cm]
學號:\idB & 姓名:\userB \\[0.3cm]
學號:\idC & 姓名:\userC \\[0.3cm]
學號:\idD & 姓名:\userD \\[0.3cm]
學號:\idE & 姓名:\userE \\[0.3cm]
\end{tabular}

\hspace{-8.4cm} \makebox[3cm][s]{所撰之專題製作報告:}\\[0.3cm]

\uline{\cTitle}  \\[0.6cm]

\hspace{-8.4cm} \makebox[3cm][s]{經本委員會審查通過} \\[0.5cm]

指導老師:\rule{0.4\textwidth}{1pt} \\[0.3cm]
審查委員:\rule{0.4\textwidth}{1pt} \\[0.3cm]
審查委員:\rule{0.4\textwidth}{1pt} \\[0.3cm]
審查委員:\rule{0.4\textwidth}{1pt} \\[0.3cm]
審查委員:\rule{0.4\textwidth}{1pt} \\[0.3cm]
系主任  :\rule{0.4\textwidth}{1pt} \\[0.3cm]

%\vspace{4cm}
% \makebox is a text box with specified width;
% option s: stretch
% use \makebox to make sure
\vfill
\makebox[10cm][s]{\Large{中華民國 \cYear 年 \cMonth 月 \cDay 日}}%
%
\end{center}

\clearpage
\normalsize


%%%%%%%%%%%%%%%%%%%%%%%%%%%%%%%
%       授權書 (計頁碼,但不印頁碼)
%%%%%%%%%%%%%%%%%%%%%%%%%%%%%%%
\thispagestyle{EmptyWaterMarkPage}  % 無 header,但在浮水印模式下會有浮水印
\phantomsection % for hyperref to register this
\addcontentsline{toc}{chapter}{使用同意書}%
\begin{center}
\fontsize{20}{14}\selectfont
% font size (relative to 12 pt):
% \large (14pt) < \Large (18pt) < \LARGE (20pt) < \huge (24pt)< \Huge (24 pt)
%
\makebox[13cm][c]{\Huge{使用同意書}}\\  %顯示中文校名
\vspace{1.5cm}
\raggedright
茲同意吾等所製之專題製作和其報告: \\[0.8cm]

\centerline{\uline{\cTitle}}


%\makebox[12cm][l]{
%\begin{minipage}
\raggedright
提供給臺北城市科技大學資管系作為資料參考、作品展示暨繼續延伸研究之用。

\centering 

%}
%\makebox[12cm][s]{\Huge{\deptCname}}\\ %顯示中文系所名
%\vspace{1.5cm}
%\makebox[8cm][s]{\Huge{102學年度專題製作}}\\
\vspace{1.5cm}
%
% Set the line spacing to single for the titles (to compress the lines)
\renewcommand{\baselinestretch}{1}   %行距 1 倍

學號:\idA 姓名:\rule{0.3\textwidth}{1pt} (簽名) \\[0.7cm]
學號:\idB 姓名:\rule{0.3\textwidth}{1pt} (簽名) \\[0.7cm]
學號:\idC 姓名:\rule{0.3\textwidth}{1pt} (簽名) \\[0.7cm]
學號:\idD 姓名:\rule{0.3\textwidth}{1pt} (簽名) \\[0.7cm]
學號:\idE 姓名:\rule{0.3\textwidth}{1pt} (簽名) \\[0.7cm]

\vspace{2cm}
指導老師:\rule{0.4\textwidth}{1pt} \\[0.3cm]


%\vspace{4cm}
% \makebox is a text box with specified width;
% option s: stretch
% use \makebox to make sure
\vfill
\makebox[10cm][s]{\Large{中華民國 \cYear 年 \cMonth 月 \cDay 日}}%
%
\end{center}
\clearpage
\normalsize


%%%%%%%%%%%%%%%%%%%%%%%%%%%%%%%
%       授權書 (計頁碼,但不印頁碼)
%%%%%%%%%%%%%%%%%%%%%%%%%%%%%%%
%
% insert the printed standard form when the thesis is ready to bind
% 在口試完成後,再將已簽名的授權書放入以便裝訂
% create an entry in table of contents for 授權書
% 目前送出空白頁
\newpage%

%\vspace{3cm}

提醒:

上一頁為「論文口試委員審定書」, 在口試完成後,再將已簽名的審定書,先掃描成電子檔,以 Adobe PDF Writer 插入置換上頁,更新電子檔,同時並放入論文紙本以便裝訂。

\vspace{2cm}

此頁為「論文授權書」,計頁碼,但不印頁碼,在口試完成後,再將已簽名的授權書之後,替換本頁以便裝訂。
{\thispagestyle{empty}%
%\phantomsection % for hyperref to register this
%\addcontentsline{toc}{chapter}{\nameCopyrightForm}%
\mbox{}\clearpage}

%%%%%%%%%%%%%%%%%%%%%%%%%%%%%%%
%       中文摘要
%%%%%%%%%%%%%%%%%%%%%%%%%%%%%%%
%
\newpage
\thispagestyle{plain}  % 無 header,但在浮水印模式下會有浮水印
% create an entry in table of contents for 中文摘要
\phantomsection % for hyperref to register this
\addcontentsline{toc}{chapter}{\nameCabstract}

% aligned to the center of the page
\begin{center}
% font size (relative to 12 pt):
% \large (14pt) < \Large (18pt) < \LARGE (20pt) < \huge (24pt)< \Huge (24 pt)
% Set the line spacing to single for the names (to compress the lines)
\renewcommand{\baselinestretch}{1}   %行距 1 倍
% it needs a font size changing command to be effective
%\large{\cTitle}\\  %中文題目
%\vspace{0.83cm}
% \makebox is a text box with specified width;
% option s: stretch
% use \makebox to make sure
% each text field occupies the same width
%\makebox[1.5cm][s]{\large{學生:}}
%\makebox[3cm][l]{\large{\myCname}} %學生中文姓名
%\hfill
%
%\makebox[3cm][s]{\large{指導教授:}}
%\makebox[3cm][l]{\large{\advisorCnameA}} \\ %教授A中文姓名
%
%\vfill
\makebox[2.5cm][s]{\large{摘要}}\\
\end{center}
% Resume the line spacing to the desired setting
\renewcommand{\baselinestretch}{\mybaselinestretch}   %恢復原設定
%it needs a font size changing command to be effective
% restore the font size to normal
\normalsize
請重寫!!!!!

雲端運算是一項新興的運算模型,可讓使用者隨時隨地透過任何連線裝置存取其應用程式。透過以使用者為主的介面,使用者可一目了然由雲端基礎架構所支援的應用程式。應用程式位在極富彈性的資料中心內,該處可動態提供並共用運算資源,以達到顯著的快速運算。拜功能強大的服務管理平台之賜,為雲端技術增加更多 資訊科技資源所需的管理成本,應該會遠低於與其他基礎架構相關的管理成本。

是什麼驅使人們使用雲端運算,原因有很多,其中包括智慧型行動裝置、高速網路連線,以及需要高密度運算與資料密集的 Web、HTML 應用程式等產品的興起。

因此,資訊科技產業的廠商紛紛投入各種雲端運算功能的研發,而企業界客戶也開始對某些層面的雲端軟體感興趣,而想要以其作為服務的主要程序與下一代的分散式運算。在虛擬化空間的領域上享有領先地位,近期更推出了支援動態基礎架構的資料中心。此版本結合了以 Web 為中心的雲端運算模型,以及最新的企業資料中心。它能透過極富彈性、多元且已虛擬化的基礎架構,為各種工作量提供以要求為導向並可動態配置的運算資源。不僅如此,它還會就安全性、資料完整性、復原與交易處理各方面進行最佳化。由於對企業資料中心與雲端運算方面都具有豐富的經驗,因此能以優異的表現,準備好為客戶提供能達到此目標的最佳解決方案。具體而言,已針對大規模的資料中心定義出雲端運算的執行架構,並持續進行強化,使其能夠執行主控各種應用程式所需的主要功能。此架構目前包括將伺服器、網路、儲存裝置、作業系統與中介軟體等項目中既複雜又耗費時間的供應程序自動化。此外,它還支援高度資料密集的工作量,並支援復原與安全方面的需求。本文將說明高階雲端運算基礎架構服務的結構與基礎技術要項,例如虛擬化、自動化、自助式入口網站、監視與容量規劃。另外,它還從以前到現在某些以此方式來建置的資料中心,討論其範例及價值論點。

關鍵字:網站應用程式(Web Application), 至少寫三個

\newpage

%%%%%%%%%%%%%%%%%%%%%%%%%%%%%%%
%       英文摘要
%%%%%%%%%%%%%%%%%%%%%%%%%%%%%%%
%
\thispagestyle{plain}  % 無 header,但在浮水印模式下會有浮水印
% create an entry in table of contents for 英文摘要
\phantomsection % for hyperref to register this
\addcontentsline{toc}{chapter}{\nameEabstract}

% aligned to the center of the page
\begin{center}
% font size:
% \large (14pt) < \Large (18pt) < \LARGE (20pt) < \huge (24pt)< \Huge (24 pt)
% Set the line spacing to single for the names (to compress the lines)
\renewcommand{\baselinestretch}{1}   %行距 1 倍
%\large % it needs a font size changing command to be effective
%\large{\eTitle}\\  %英文題目
%\vspace{0.83cm}
% \makebox is a text box with specified width;
% option s: stretch
% use \makebox to make sure
% each text field occupies the same width
%\makebox[2cm][s]{\large{Student: }}
%\makebox[5cm][l]{\large{\myEname}} %學生英文姓名
%\hfill
\large{ABSTRACT}\\
%\vspace{0.5cm}
\end{center}
% Resume the line spacing the desired setting
\renewcommand{\baselinestretch}{\mybaselinestretch}   %恢復原設定
%\large %it needs a font size changing command to be effective
% restore the font size to normal
\normalsize
%%%%%%%%%%%%%
請重寫!!!!!

The high in the clouds operation is an emerging operation model,may let the user penetrate any segment installment to deposit and withdraw its application formula anytime and anywhere.The penetration by the user primarily interface,the user may be clear at a glance the application formula which supports by the high in the clouds foundation construction.The application formula position in the extremely rich elastic material center,this place may dynamic provide and use in common the operation resources,achieves the remarkable fast operation.Does obeisance the function formidable service to manage bestowing of the platform, increases the management cost for the high in the clouds technology which the more information science and technology resources need, should be able to be lower than far with other foundation construction correlation management cost.

  Is any urges the people to use the high in the clouds operation, the reason has very many, including the wisdom running gear,the high speed network segment, as well as needs the high density operation and material crowded Web,product and so on Html application formula starting.

    Therefore, the information science and technology industry manufacturer invests each kind of high in the clouds operation function in abundance the research and development,but the business community customer also starts to certain stratification plane high in the clouds software to be interested, but wants by it to take the service the main program and the next generation disperser -like operation.Enjoys the leading status in the virtualization space domain,has in the near future promoted the support dynamic foundation construction material center.
   
   This edition unified take Web as the central high in the clouds operation model,as well as newest enterprise material center.It can penetrate the extremely rich elasticity,multi-dimensional also already the virtualization foundation construction,provides for each kind of work load take requests as the guidance and may the dynamic disposition operation resources.Not only that, it also can on the security,the material integrity,the restoration and the transaction processes various aspects to carry on the optimization. Because all has the rich experience to the enterprise material center and the high in the clouds operation aspect,therefore can take the outstanding performance, prepare to provide can achieve this goal as the customer the best solution.Says specifically,has defined the high in the clouds operation in view of the large-scale material center the execution construction, and continues to carry on the strengthening,causes main function which its can  the execution master control each kind of application formula need.This construction at present including projects in and so on server, network,storage installment,work systematic and intermediary software both complex and consumes the time the supply automatic programming.In addition, it also supports highly the material crowded work load, and supports the restoration and the security aspect demand.This article will show the higher order high in the clouds operation foundation construction service the structure and the foundation technology important item,for example virtualization, automated, self-service type entrance website, surveillance and capacity planning.Moreover,it also from before to the present certain the material center which establishes by this way, discusses its model and the value argument.

Keywords: Internet applications (Web Application), 至少寫三個

\newpage

%%%%%%%%%%%%%%%%%%%%%%%%%%%%%%%
%       誌謝
%%%%%%%%%%%%%%%%%%%%%%%%%%%%%%%
%
% Acknowledgment
\chapter*{\protect\makebox[5cm][s]{\nameAckn}} %\makebox{} is fragile; need protect
\phantomsection % for hyperref to register this
\addcontentsline{toc}{chapter}{\nameAckn}
畢業論文最終能夠順利完成,在此我們要感謝許多長輩們的協助。首先,最以誠摯的心意感謝我的指導教授 xxx 博士這些年來的親切指導與諄諄教誨;每當遇到問題時,老師總是會耐心地為我指點迷津,充實了我們的學術涵養,讓我們可以從中學習到非常大量的做研究之方法以及該有的態度。此外,還要特別感謝 .......。

能夠進入臺北城市科技大學資訊管理系四年的大學生活,我們真的非常幸運也非常珍惜,隨著畢業論文的完成,在求學期間,回憶這些年來與同學們、老師們與朋友們的點點滴滴,有汗水,有淚光,很高興能夠擁有這段特別的回憶,讓我們經歷了一段不一樣的特殊經驗。

最後,我們將論文獻給我們的家人以及所有關懷我們、給予我們寶貴意見的朋友們,感謝你們的支持與鼓勵,讓我能夠順利完成本論文!感謝您們。

於臺北城市科技大學 資訊管理系

2013 年 6 月

\newpage

%%%%%%%%%%%%%%%%%%%%%%%%%%%%%%%
%       目錄
%%%%%%%%%%%%%%%%%%%%%%%%%%%%%%%
%
% Table of contents
\renewcommand{\contentsname}{\protect\makebox[5cm][s]{\nameToc}}

%\makebox{} is fragile; need protect
\phantomsection % for hyperref to register this
\addcontentsline{toc}{chapter}{\nameToc}
\tableofcontents

%%%%%%%%%%%%%%%%%%%%%%%%%%%%%%%
%       表目錄
%%%%%%%%%%%%%%%%%%%%%%%%%%%%%%%
%
% List of Tables
\newpage
\renewcommand{\listtablename}{\protect\makebox[5cm][s]{\nameLot}}
%\makebox{} is fragile; need protect
\phantomsection % for hyperref to register this
\addcontentsline{toc}{chapter}{\nameLot}
%\renewcommand{\numberline}[1]{\loflabel~#1\hspace*{1em}}
\renewcommand{\numberline}[1]{\lotlabel~#1\hspace*{1em}}
\listoftables

%%%%%%%%%%%%%%%%%%%%%%%%%%%%%%%
%       圖目錄
%%%%%%%%%%%%%%%%%%%%%%%%%%%%%%%
%
% List of Figures
\newpage
\renewcommand{\listfigurename}{\protect\makebox[5cm][s]{\nameTof}}
%\makebox{} is fragile; need protect
\phantomsection % for hyperref to register this
\addcontentsline{toc}{chapter}{\nameTof}
\renewcommand{\numberline}[1]{\loflabel~#1\hspace*{1em}}
%\renewcommand{\numberline}[1]{\lotlabel~#1\hspace*{1em}}
\listoffigures

%%%%%%%%%%%%%%%%%%%%%%%%%%%%%%%
%       程式碼 lstlistoflistings 目錄
%%%%%%%%%%%%%%%%%%%%%%%%%%%%%%%
%
% List of Figures
\newpage
\renewcommand{\lstlistlistingname}{\protect\makebox[5cm][s]{程式碼目錄}}
%\makebox{} is fragile; need protect
\phantomsection % for hyperref to register this
\addcontentsline{toc}{chapter}{程式碼目錄}
\renewcommand{\numberline}[1]{程式碼~#1\hspace*{1em}}
%\renewcommand{\numberline}[1]{\lotlabel~#1\hspace*{1em}}
\lstlistoflistings

\newpage
%% 論文本體頁碼回復為阿拉伯數字計頁,並從頭起算
\pagenumbering{arabic}
%%%%%%%%%%%%%%%%%%%%%%%%%%%%%%%%
