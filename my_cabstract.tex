請重寫!!!!!

雲端運算是一項新興的運算模型,可讓使用者隨時隨地透過任何連線裝置存取其應用程式。透過以使用者為主的介面,使用者可一目了然由雲端基礎架構所支援的應用程式。應用程式位在極富彈性的資料中心內,該處可動態提供並共用運算資源,以達到顯著的快速運算。拜功能強大的服務管理平台之賜,為雲端技術增加更多 資訊科技資源所需的管理成本,應該會遠低於與其他基礎架構相關的管理成本。

是什麼驅使人們使用雲端運算,原因有很多,其中包括智慧型行動裝置、高速網路連線,以及需要高密度運算與資料密集的 Web、HTML 應用程式等產品的興起。

因此,資訊科技產業的廠商紛紛投入各種雲端運算功能的研發,而企業界客戶也開始對某些層面的雲端軟體感興趣,而想要以其作為服務的主要程序與下一代的分散式運算。在虛擬化空間的領域上享有領先地位,近期更推出了支援動態基礎架構的資料中心。此版本結合了以 Web 為中心的雲端運算模型,以及最新的企業資料中心。它能透過極富彈性、多元且已虛擬化的基礎架構,為各種工作量提供以要求為導向並可動態配置的運算資源。不僅如此,它還會就安全性、資料完整性、復原與交易處理各方面進行最佳化。由於對企業資料中心與雲端運算方面都具有豐富的經驗,因此能以優異的表現,準備好為客戶提供能達到此目標的最佳解決方案。具體而言,已針對大規模的資料中心定義出雲端運算的執行架構,並持續進行強化,使其能夠執行主控各種應用程式所需的主要功能。此架構目前包括將伺服器、網路、儲存裝置、作業系統與中介軟體等項目中既複雜又耗費時間的供應程序自動化。此外,它還支援高度資料密集的工作量,並支援復原與安全方面的需求。本文將說明高階雲端運算基礎架構服務的結構與基礎技術要項,例如虛擬化、自動化、自助式入口網站、監視與容量規劃。另外,它還從以前到現在某些以此方式來建置的資料中心,討論其範例及價值論點。

關鍵字:網站應用程式(Web Application), 至少寫三個
